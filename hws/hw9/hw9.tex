\input{settings} % add packages, settings, and declarations in settings.tex
\usepackage{url}
\usepackage{amsmath,amssymb,amsthm}


\newcommand{\Prob}[1]{{{\bf{Pr}}\left[{#1}\right]}}
\newcommand{\Mean}[1]{{\mathbb E}\left[{#1}\right]}
\newcommand{\Var}[1]{{\mathbb Var}\left[{#1}\right]}


\begin{document}

\lhead{Prof. C.E. Tsourakakis} 
\rhead{CS365 Spring '22 \\ Foundations of Data Science \\ Assignment 9} 
\cfoot{\thepage\ of \pageref{LastPage}}


\section*{Instructions}
\framebox{%
	\begin{minipage}{0.9\linewidth}
		\begin{itemize}
			\item The homework is due on \underline{{\bf Friday 4/22 at 5pm ET}}.  
			\item There are 2 problems in total.
			\item No extension will be provided, unless for serious documented reasons.
			\item {\bf Start early!}
			\item Study the material taught in class, and feel free to do so in small groups, but the solutions should be a product of your own work. 
			\item This  is not a multiple choice homework;   reasoning, and mathematical proofs are required before giving your final answer.
		\end{itemize}
\end{minipage}}

\section{Short proofs [50 points]} 

\begin{enumerate}
	\item [(a)] [10 pts] Given a square matrix $A^{n \times n}$ and a constant $\kappa$ we define the shifted matrix $A-\kappa I$ where $I^{n \times n}$ is the identity matrix. Prove that if $\lambda$ is an eigenvalue of $A$, then $\lambda-\kappa$ is an eigenvalue of $A -\kappa I$. 
	
	\vspace{2mm} 
	\item[(b)]  [10 pts] Consider the (undirected) clique graph $K_n$ on $n$ nodes, where every two distinct vertices in the clique are adjacent.  Compute analytically the eigenvalues of the adjacency matrix. 	
	\vspace{2mm} 
	
	{\it Hint: }  Shift the adjacency matrix with an appropriate value $\kappa$ to obtain a matrix who eigenvalues are easy to calculate, and then shift back to the original matrix using (a).
	

	\item[(c)]  [10 pts]  Consider a random walk on a connected undirected graph with $n$ nodes and $m$ edges. Prove that the stationary distribution $\pi=(\pi_1,\ldots,\pi_n)$ satisfies
	$\pi_j = \frac{deg(j)}{2m}$ for all nodes $j \in [n]$. 
	
	
	\item[(d)]  [10 pts]   Can a Markov chain have infinite stationary distributions? Explain your answer.
	
		\item [(e) ]  [10 pts]
	What is the expected number of coin tosses to obtain heads-tails-heads (HTH) consecutively using a fair coin? \\ 


 

\end{enumerate}   

 

\section{HITS algorithm [50 points]}

For the programming assignment see the Jupyter notebook in our Github page \url{https://github.com/tsourolampis/cs365-spring22} under the HW directory.

 
 

\end{document}
